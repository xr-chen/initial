\documentclass[10pt]{beamer}
\usepackage[noindent]{ctexcap}
\usetheme{metropolis}
\usepackage{appendixnumberbeamer}

\usepackage{booktabs}
\usepackage[scale=2]{ccicons}
\usepackage{amsthm}
\usepackage{pgfplots}
\usepgfplotslibrary{dateplot}

\usepackage{xspace}
\newcommand{\themename}{\textbf{\textsc{metropolis}}\xspace}
\newtheorem{theo}{定理}

\title{P91定理}
\subtitle{}
\date{\today}
\author{07117117陈兴嵘}
\institute{}
% \titlegraphic{\hfill\includegraphics[height=1.5cm]{logo.pdf}}

\begin{document}

\maketitle

\begin{frame}{目录}
  \setbeamertemplate{section in toc}[sections numbered]
  \tableofcontents[hideallsubsections]
\end{frame}

\section{定理叙述}

\begin{frame}[fragile]{定理叙述}
\begin{theo}
  在$C[0,1]$中处处不可微的函数集合$E$是非空的,更确切地,$E$的余集是第一纲集\cite{Knuth92}
\end{theo}
\end{frame}
\section{定理背景}
\begin{frame}[fragile]{定理背景}
Weierstrass构造出了一个处处连续却处处不可微的函数
\begin{equation}\label{}
  W(x)=\sum_{n=0}^{\infty}a^n\cos(b^n\pi x)
\end{equation}
其中$0<a<1$,$b$为正奇数,且:
\begin{equation}\label{}
  ab>1+\frac{3}{2}\pi
\end{equation}
\end{frame}

\begin{frame}[fragile]{连续性}
\begin{proof}
  函数列$a^n\cos(b^n\pi x)$满足:
  \begin{equation}\label{}
    |a^n\cos(b^n\pi x)|\leq a^n
  \end{equation}
  且根据a的定义,正项级数$\sum_{n=0}^{+\infty}a^n$是收敛的,由Weierstrass判别法知,这个函数项级数一致收敛,且每一个函数项都是$\mathbb{R}$上的连续函数,则$W(x)$在$\mathbb{R}$上连续
\end{proof}
\end{frame}
\begin{frame}[fragile]{不可导性}
思路:
\begin{proof}
  对于任意点$x\in\mathbb{R}$都能找出趋于$x$的两个不同数列$\{x_n\}$和$\{x_n^{'}\}$使得:
  \begin{equation}\label{}
    \varliminf_{n\to\infty}\frac{W(x_n)-W(x)}{x_n-x}>\varlimsup_{n\to\infty}\frac{W(x^{'}_n)-W(x)}{x^{'}_n-x}
  \end{equation}
  与函数可导的定义矛盾。
\end{proof}
\end{frame}



\section{证明思路}

\begin{frame}{证明思路}
	\begin{itemize}[<+- | alert@+>]
%    \item \alert<4>{This is\only<4>{ really} important}
    \item 设全空间$\mathcal{X}$为$C[0,1]$
    \item 利用可导的性质来定义集合$A_n$找到和处处不可微函数集$E$的关系
    \item 证明$A_n$是闭集
    \item 证明$A_n$没有内点得到每个$A_n$是疏集
    \item 由Baire定理推出$\mathcal{X}$是第二纲集,从而$E$也是第二纲集
    \item
  \end{itemize}
\end{frame}

\section{证明过程}

\begin{frame}[fragile]{Step1}
\textbf{证明}\\
取$\mathcal{X}=C[0,1]$,设$A_n$表示$\mathcal{X}$中这样一些元素$f$之集。对$f,\exists s\in[0,1]$,使对适合的$0\leq s+h\leq 1$与$|h|\leq 1/n$的任何$h$下成立
\begin{equation}\label{}
  |\frac{f(s+h)-f(s)}{h}|\leq n
\end{equation}
若$f$在某个点$s$处可微,则必有正整数$n$,使得$f\in A_n$
\begin{equation}\label{236}
  \mathcal{X}\setminus E\subset \bigcup_{n=1}^{\infty}A_n
\end{equation}
\end{frame}

\begin{frame}[fragile]{Step2}
\onslide<1->{证明$A_n$是闭的,若$f\in\mathcal{X}\setminus A_n$,则$\forall s\in[0,1],\exists h_s,$(任意给定的s)使得
\begin{equation}\label{qiqi}
  |h_s|\leq \frac{1}{n}\quad\mbox{且}\quad|f(s+h_s)-f(s)|>n|h_s|
\end{equation}
由$f$的连续性,$\exists\epsilon_s>0$,以及$s$的某个适当的邻域$J_s$,对$\forall\sigma\in J_s$有
\begin{equation}\label{shuo}
  |f(\sigma+h_s)-f(\sigma)|>n|h_s|+2\epsilon_s
\end{equation}}
\onslide<2->{
因为
\begin{equation}\label{}
\begin{split}
    &  |f(\sigma+h_s)-f(\sigma)|\\
     =& |f(\sigma+h_s)-f(s+h_s)+f(s+h_s)-f(s)+f(s)-f(\sigma)|
\end{split}
\end{equation}
由\eqref{qiqi}知,当$f(s+h_s)-f(s)>0$时,找到邻域$J_s$满足
\begin{equation}
f(\sigma+h_s)-f(s+h_s)+f(s)-f(\sigma)>2\epsilon_s>0
\end{equation}
从而对$\epsilon_s$满足\eqref{shuo}式,当$f(s+h_s)-f(s)<0$时(\eqref{qiqi}式说明没有等于零的情形),同理可以找到合适的邻域$J_s$对$\epsilon_s$满足\eqref{shuo}}
\end{frame}

\begin{frame}[fragile]{Step2}
\onslide<1->{对于$\forall s\in[0,1]$都满足\eqref{shuo},那么$J_{s_i}$一定可以覆盖区间$[0,1]$根据有限覆盖定理,可设$J_{s_1}...J_{s_m}$覆盖了$[0,1]$,设
\begin{equation}\label{}
  \epsilon=\min\{\epsilon_{s_1},\epsilon_{s_2},...,\epsilon_{s_m}\}
\end{equation}}\onslide<2->{
若$g\in\mathcal{X}$适合$||g-f||<\epsilon$,则由\eqref{shuo}式,对$\forall\sigma\in J_{s_k}(k=1,...m)$有
\begin{equation}\label{}
\begin{split}
&|g(\sigma+h_{s_k})-g(\sigma)|\\
=&|g(\sigma+h_{s_k})-f(\sigma+h_{s_k})+f(\sigma+h_{s_k})-f(\sigma)+f(\sigma)-g(\sigma)|\\
\geq&\bigg||f(\sigma+h_{s_k})-f(\sigma)|-|f(\sigma)-g(\sigma)|-|g(\sigma+h_{s_k})-f(\sigma+h_{s_k})|\bigg|\\
\geq&\bigg||f(\sigma+h_{s_k})-f(\sigma)|-2||g-f||\bigg|\\
\geq&\bigg||f(\sigma+h_{s_k})-f(\sigma)|-2\epsilon\bigg|>n|h_{s_k}|
\end{split}
\end{equation}}\onslide<3->{
根据\eqref{qiqi}的定义$g\in\mathcal{X}\setminus A_n$,从而$f$是$\mathcal{X}\setminus A_n$的内点,由$f$的任意性可得,$\mathcal{X}\setminus A_n$是开集,从而$A_n$是闭集}
\end{frame}

\begin{frame}[fragile]{Step3}
\onslide<1->{$A_n$是闭集,因此$A_n=\overline{A_n}$,下面证明$A_n$没有内点,$\forall f\in A_n$,$\forall\epsilon>0$,由Weierstrass逼近定理,存在多项式$p$,使得
\begin{equation}\label{}
  ||f-p||<\frac{\epsilon}{2}
\end{equation}}\onslide<2->{
(Weierstrass逼近定理:闭区间上的连续函数可用多项式级数一致逼近)}\onslide<3->{
由闭区间连续函数的性质,可以得到$p$的导数在$[0,1]$上有界,因此根据中值定理$\exists M>0$使得$\forall s\in[0,1]$及$|h|<1/n$有
\begin{equation}\label{}
  |p(s+h)-p(s)|\leq M|h|
\end{equation}}\onslide<4->{
设$g(s)\in C[0,1]$是一个分段线性函数,满足$||g||<\epsilon/2$,并且各段的斜率的绝对值大于$M+n$,从而由\eqref{qiqi}可得
\begin{equation}\label{}
  p+g\in B(f,\epsilon),\quad\mbox{且}\quad p+g\notin A_n
\end{equation}
由$\epsilon$的任意性,可得$A_n$没有内点,是疏集,从而$\cup_{n=1}^{\infty}A_n$是第一纲集,$\mathcal{X}$完备,由Baire定理,$\mathcal{X}$是第二纲集,根据\eqref{236}得,$E$也是第二纲集}
\end{frame}

\begin{frame}[standout]
\Huge{感谢聆听}
\end{frame}

\appendix

\begin{frame}[allowframebreaks]{References}
  \bibliographystyle{plain}
  \bibliography{demo}


\end{frame}

\end{document}
